\documentclass[a4paper]{article}
\usepackage{makeidx}
\usepackage{ifthen}
\usepackage{alltt} 
\usepackage{hyperref} 

\title{ncWMS Developer's Guide}
\author{Guy Griffiths}
\makeindex

\begin{document}

\maketitle

%\begin{abstract}
%This document describes the layout of the code and how to use ncWMS as a basis
%for another custom WMS.  It is currently rather basic.
%\end{abstract}

%\tableofcontents

%\newpage

\section{Intro}
{\tt edal-ncwms} is written in such a way as to be as easy as possible to
extend.  However, some familiarity with the code is necessary for developers to
make useful extensions.  This guide gives an overview of the code structure and
the terminology used, as well as how to use the Maven archetype supplied with
the EDAL libraries to create a new webapp based on {\tt edal-ncwms}.

\section{Code Structure}

\subsection{Core data types and their imlpementations}

\subsection{Graphics library}

\subsection{The Unidata CDM}

\subsection{ncWMS}

\subsection{}


\section{Creating New Projects Based on ncWMS}
The EDAL libraries contain a Maven archetype, named
{\tt edal-ncwms-based-webapp}.  Using this archetype is the recommended
way of basing a new webapp on {\tt edal-ncwms}.  A skeletal project
implementation can be created using the command:
\begin{alltt}
mvn archetype:generate -DarchetypeGroupId=uk.ac.rdg.resc
-DarchetypeArtifactId=edal-ncwms-based-webapp
-DarchetypeVersion=\emph{<edal-version>} -DgroupId=\emph{<groupId>}
-DartifactId=\emph{<artifactId>}
\end{alltt}
This creates an application which contains all the functionality of ncWMS, with
one additional sample method, illustrating how to hook into the ncWMS datasets
and display the configured features and their members.  The files created and
their roles are discussed below:

\subsection{pom.xml}
This defines a basic Maven {\tt pom.xml} which imports the
version of {\tt edal-ncwms} matching the {\tt <edal-version>} you specified
during project creation (i.e. the version of the {\tt edal-ncwms-based-webapp}
archetype).  It imports {\tt edal-ncwms} as an overlay, meaning that the entire
contents of the WAR file are imported \emph{unless they overwrite an existing
file}.  More information on Mavem overlays can be found at
\url{http://maven.apache.org/plugins/maven-war-plugin/overlays.html}
  
\subsection{src/webapp/WEB-INF/WMS-servlet.xml}
This is already defined in {\tt edal-ncwms} and handles the configuration of
the WMS servlet.  In {\tt edal-ncwms} it simply points at a file named
{\tt ncWMS-beans.xml} which contains the actual configuration.  Here it points
at both {\tt ncWMS-beans.xml} and {\tt app-beans.xml}.  This allows developers
to put all of their application-specific configuration into {\tt app-beans.xml}
whilst still retaining all of the {\tt edal-ncwms} functionality.

\subsection{src/webapp/WEB-INF/app-beans.xml}
This handles application-specific configuration.  It is handled by
\href{http://www.springsource.org/}{Spring}, and familiarity with Spring is
assumed.  {\tt app-beans.xml} contains a single Spring bean which defines a
new controller, some URL mappings, and a redefinition of the ncwmsContext bean,
which overwrites the default configuration directory.

New controllers can be defined and mapped to URLs in here.

\subsection{src/webapp/WEB-INF/web.xml}
This is the main entry-point of a webapp.  Here it is defined to import from two
existing XML files: {\tt ncwms-web.xml} and {\tt app-web.xml}.  The situation
here is very similar to that of {\tt WMS-servlet.xml} - all
application-specific configuration normally handled by {\tt web.xml} should go
into {\tt app-web.xml}

\subsection{src/webapp/WEB-INF/app-web.xml}
This is the place to put any configuration which would normally go into
{\tt web.xml}.  This file is empty, since no configuration is needed to write a
new application based on ncWMS.  The existing {\tt edal-ncwms} code defines a
URL mapping such that any request of the form
{\tt http://server/webapp/vendor/**/*} is mapped to the WMS servlet.  Arbitrary
mappings can be added by mapping the {\tt WMS} servlet to the desired path: 
\begin{alltt}
<servlet-mapping>
    <servlet-name>WMS</servlet-name>
    <url-pattern>/path/*</url-pattern>
</servlet-mapping>
\end{alltt}

\subsection{src/main/package/AppController.java}
This is where the method that actually does the work is defined.  This is a
single method called {\tt doTest}.  Note that the {\tt AppController} class has
access to the ncWMS {\tt Config} object, and as such can access any datasets
defined in ncWMS.  The {\tt doTest} method simply gets all datasets and loops
through the contained features and their members, printing them to the output
stream.  Whilst simple, this example should provide a good starting point for
developing new applications with {\tt edal-ncwms}.

\subsection{src/test/package/AppTest.java}
This is a standard test suite skeleton, using junit. 

\subsection{Using the example application}
Once a skeletal implementation of a project has been created, it can be tested
immediately.  By invoking {\tt mvn package}, the new webapp will be packaged
into {\tt target/appname-version.war}.  Once this is deployed to a suitable
application server, it will behave exactly like {\tt edal-ncwms}, with one
exception - there is a method bound to {\tt http://server/webapp/vendor/test}
which will execute the code found in {\tt AppController.java}.  By default this
will simply display a list of all configured datasets, their features, and the
names of the members of each feature.

\end{document}